\documentclass[12pt,a4paper]{article}

\usepackage[utf8]{inputenc}
\usepackage[T1]{fontenc}
\usepackage[polish]{babel}
\usepackage{helvet}
\renewcommand{\familydefault}{\sfdefault}
\usepackage{geometry}
\usepackage{graphicx}
\usepackage{hyperref}
\usepackage{listings}
\usepackage{float}
\usepackage{caption}
\usepackage{setspace}

\geometry{margin=2.5cm}
\onehalfspacing


\title{
\textbf{Raport z realizacji projektu programistycznego} \\[0.5cm]
\large System biblioteczny
}

\author{
Autorzy (grupa C): \\[0.2cm]
Dawid Błaszczyk - 280518 \\
Błazej Kowal - 280655 \\
Alina Lenart - 280588 \\
Bartosz Wacławiak - 280462 \\[0.4cm]
Prowadzący laboratorium: \\
dr inż. Krzysztof Chudzik
}

\date{
Data ukończenia pracy: \\ 
07.01.2025
}

\begin{document}

\maketitle
\thispagestyle{empty}
\newpage

% Spis treści
\tableofcontents
\newpage

\section{Wymagania projektowe}

\subsection{Wymagania funkcjonalne}

System biblioteczny IoT musi spełniać następujące wymagania funkcjonalne:

\subsubsection{Obsługa kart RFID}
\begin{itemize}
    \item System musi umożliwiać odczyt kart RFID za pomocą czytnika MFRC522.
    \item System musi automatycznie wykrywać przyłożenie i zabranie karty RFID.
    \item Odczytany identyfikator karty (UID) musi być konwertowany do formatu szesnastkowego i przesyłany do serwera centralnego.
    \item System musi rozróżniać między kartami klientów biblioteki i kartami przypisanymi do książek.
\end{itemize}

\subsubsection{Zarządzanie bazą danych}
\begin{itemize}
    \item System musi przechowywać informacje o klientach (imię, nazwisko, powiązana karta RFID).
    \item System musi przechowywać informacje o książkach (tytuł, autor, powiązana karta RFID).
    \item System musi rejestrować wypożyczenia i zwroty książek z datami operacji.
    \item System musi umożliwiać automatyczne tworzenie nowych rekordów kart przy pierwszym skanowaniu nieznanego UID.
\end{itemize}

\subsubsection{Proces wypożyczania i zwracania}
\begin{itemize}
    \item System musi umożliwiać wypożyczenie książki poprzez zeskanowanie karty klienta, a następnie karty książki.
    \item System musi umożliwiać zwrot książki poprzez analogiczny proces skanowania.
    \item System musi wyświetlać informacje o aktywnych wypożyczeniach klienta po zeskanowaniu jego karty.
    \item System musi weryfikować poprawność operacji (np. czy książka jest dostępna, czy klient już ją wypożyczył).
\end{itemize}

\subsubsection{Interfejs użytkownika}
\begin{itemize}
    \item System musi posiadać webowy interfejs graficzny dostępny przez przeglądarkę.
    \item Interfejs musi umożliwiać przeglądanie listy wszystkich klientów, książek i wypożyczeń.
    \item Interfejs musi umożliwiać ręczne dodawanie, edycję i usuwanie klientów oraz książek.
    \item Interfejs musi wyświetlać informacje o zeskanowanej karcie w czasie rzeczywistym.
    \item Interfejs musi umożliwiać przeprowadzenie pełnego procesu wypożyczania/zwrotu z graficznym przewodnikiem.
\end{itemize}

\subsubsection{Komunikacja MQTT}
\begin{itemize}
    \item Terminale RFID (Raspberry Pi) muszą komunikować się z serwerem centralnym przez protokół MQTT.
    \item System musi publikować zdarzenia skanowania kart na topic \texttt{raspberry/rfid/scan}.
    \item Serwer musi odpowiadać z danymi o kliencie lub książce na topic \texttt{raspberry/rfid/response}.
    \item System musi obsługiwać sterowanie diodami LED przez MQTT (topic \texttt{raspberry/led}).
\end{itemize}

\subsubsection{Informacje zwrotne dla użytkownika terminala}
\begin{itemize}
    \item System musi sygnalizować gotowość do skanowania zieloną diodą LED.
    \item System musi sygnalizować przetwarzanie karty czerwoną diodą LED.
    \item System musi emitować dźwięk buzzera po pomyślnym odczytaniu karty.
    \item System musi wyświetlać informacje o stanie operacji na wyświetlaczu OLED (oczekiwanie, wykryto kartę, przetwarzanie, dane klienta/książki).
\end{itemize}

\subsection{Wymagania niefunkcjonalne}

System musi spełniać następujące wymagania niefunkcjonalne:

\subsubsection{Wydajność}
\begin{itemize}
    \item Czas odpowiedzi serwera na żądanie API nie powinien przekraczać 500ms w warunkach normalnego obciążenia.
    \item System musi przetwarzać zdarzenia RFID w czasie rzeczywistym (opóźnienie poniżej 1 sekundy od momentu skanowania do wyświetlenia informacji).
    \item Aplikacja webowa musi ładować się w czasie nie dłuższym niż 3 sekundy przy standardowym połączeniu internetowym.
\end{itemize}

\subsubsection{Niezawodność i stabilność}
\begin{itemize}
    \item System musi być odporny na tymczasową utratę połączenia z brokerem MQTT i automatycznie wznawiać komunikację.
    \item W przypadku błędu odczytu karty RFID, system musi wyświetlić komunikat o błędzie i umożliwić ponowną próbę.
    \item Baza danych musi zapewniać integralność danych (brak duplikatów wypożyczeń, prawidłowe daty operacji).
\end{itemize}

\subsubsection{Skalowalność}
\begin{itemize}
    \item Architektura systemu musi umożliwiać łatwe dodanie kolejnych terminali RFID bez modyfikacji kodu serwera.
    \item Baza danych musi być zaprojektowana w sposób umożliwiający przechowywanie tysięcy rekordów klientów i książek.
\end{itemize}

\subsubsection{Bezpieczeństwo}
\begin{itemize}
    \item Dane w bazie danych muszą być zabezpieczone przed nieautoryzowanym dostępem poprzez odpowiednią konfigurację uprawnień.
    \item System musi walidować wszystkie dane wejściowe z API, aby zapobiec nieprawidłowym operacjom.
\end{itemize}

\subsubsection{Środowisko uruchomieniowe i przenośność}
\begin{itemize}
    \item Terminal RFID musi działać na platformie Raspberry Pi 4B z systemem operacyjnym Raspberry Pi OS.
    \item Serwer backendowy musi być uruchamialny na systemach Linux, macOS i Windows.
    \item Aplikacja webowa musi być kompatybilna z nowoczesnymi przeglądarkami (Chrome, Firefox, Safari, Edge).
    \item System musi umożliwiać uruchomienie brokera MQTT w sieci lokalnej (np. na Raspberry Pi lub innym serwerze).
    \item Baza danych SQLite musi być przenośna i nie wymagać dodatkowej konfiguracji serwera baz danych.
\end{itemize}

\subsubsection{Technologie i standardy}
\begin{itemize}
    \item Backend: Node.js z frameworkiem NestJS, TypeScript, TypeORM.
    \item Frontend: React 19 z TypeScript, React Router, Tailwind CSS, Vite.
    \item Komunikacja IoT: Protokół MQTT z brokerem uruchomionym w sieci lokalnej.
    \item Baza danych: SQLite 3.
    \item Raspberry Pi: Python 3 z bibliotekami paho-mqtt (klient MQTT) i mfrc522 (obsługa czytnika RFID).
    \item Komunikacja w czasie rzeczywistym: WebSockets (Socket.IO) dla aktualizacji interfejsu użytkownika.
\end{itemize}

\subsubsection{Użyteczność}
\begin{itemize}
    \item Interfejs graficzny musi być intuicyjny i nie wymagać specjalistycznego przeszkolenia.
    \item System musi dostarczać jasne komunikaty o stanie operacji (powodzenie, błąd, oczekiwanie).
    \item Wyświetlacz OLED na terminalu RFID musi prezentować czytelne informacje o aktualnym stanie systemu.
    \item Kolorowe diody LED muszą jednoznacznie sygnalizować stan systemu (zielony = gotowy, czerwony = przetwarzanie/błąd).
\end{itemize}

\subsubsection{Dokumentacja i kod}
\begin{itemize}
    \item Kod źródłowy musi być czytelny i zgodny ze standardami danego języka programowania.
    \item Projekt musi zawierać pliki konfiguracyjne umożliwiające łatwe uruchomienie systemu.
\end{itemize}


\section{Opis architektury systemu}

System został zaprojektowany w architekturze wielowarstwowej. Składa się z następujących elementów:
\begin{itemize}
    \item warstwy klienckiej (aplikacja webowa React),
    \item warstwy serwerowej (backend NestJS),
    \item warstwy komunikacyjnej (MQTT, WebSockets),
    \item warstwy danych (baza danych SQLite),
    \item warstwy urządzeń IoT (Raspberry Pi z czytnikiem RFID).
\end{itemize}

\subsection{Schemat architektury aplikacji}
Poniżej przedstawiono schemat architektury systemu z uwzględnieniem architektury sieciowej.

Architektura systemu opiera się na następujących komponentach:
\begin{itemize}
    \item \textbf{Frontend (React)} - aplikacja webowa działająca w przeglądarce, komunikująca się z backendem przez REST API i WebSockets (Socket.IO).
    \item \textbf{Backend (NestJS)} - serwer aplikacyjny obsługujący logikę biznesową, komunikujący się z bazą danych SQLite oraz brokerem MQTT.
    \item \textbf{Broker MQTT} - broker komunikacji MQTT uruchomiony w sieci lokalnej (np. na Raspberry Pi lub osobnym serwerze), pośredniczący w komunikacji między terminalem RFID a backendem.
    \item \textbf{Raspberry Pi} - terminal RFID wyposażony w czytnik MFRC522, diody LED, buzzer i wyświetlacz OLED, komunikujący się z backendem przez protokół MQTT.
    \item \textbf{Baza danych SQLite} - lokalna baza danych przechowująca informacje o klientach, książkach, kartach RFID i wypożyczeniach.
\end{itemize}

Komunikacja w systemie odbywa się następująco:
\begin{itemize}
    \item Raspberry Pi publikuje zdarzenia skanowania kart na topic \texttt{raspberry/rfid/scan}.
    \item Backend subskrybuje ten topic, przetwarza dane i odpowiada na topic \texttt{raspberry/rfid/response}.
    \item Backend steruje diodami LED na Raspberry Pi przez topic \texttt{raspberry/led}.
    \item Frontend otrzymuje aktualizacje w czasie rzeczywistym przez WebSockets (Socket.IO).
\end{itemize}
\begin{figure}[H]
    \centering
    \includegraphics[width=1\linewidth]{ArchitekturaIOT.drawio.png}
    \caption{Schemat architektury systemu}
    \label{fig:architectue}
\end{figure}

\section{Opis działania i prezentacja interfejsu}

\subsection{Prezentacja interfejsu użytkownika}

Aplikacja webowa oferuje następujące funkcjonalności:

\subsubsection{Strona główna}

\begin{figure}[H]
    \centering
    \includegraphics[width=1\linewidth]{Zrzut ekranu z 2026-01-12 12-17-02.png}
    \caption{Strona główna}
    \label{fig:home}
\end{figure}

Strona główna pełni funkcję centrum nawigacyjnego systemu. Zawiera sekcję hero z tytułem i opisem systemu, dwie główne karty akcji umożliwiające skanowanie kart RFID użytkowników i książek, sekcję szybkiego dostępu z przyciskami prowadzącymi do listy użytkowników i książek, oraz sekcję informacyjną "Jak to działa?" wyjaśniającą proces korzystania z systemu w trzech krokach. Po kliknięciu karty skanowania, wyświetla się dialog z instrukcjami, a system oczekuje na przyłożenie karty RFID do czytnika.

\subsubsection{Zarządzanie klientami}

\begin{figure}[H]
    \centering
    \includegraphics[width=1\linewidth]{Zrzut ekranu z 2026-01-12 12-18-54.png}
    \caption{Lista klientów}
    \label{fig:clients}
\end{figure}

Strona \texttt{/clients} umożliwia:
\begin{itemize}
    \item Przeglądanie listy wszystkich klientów biblioteki.
    \item Dodawanie nowych klientów (imię, nazwisko, email).
    \item Edycję danych istniejących klientów.
    \item Usuwanie klientów.
    \item Przypisywanie kart RFID do klientów.
    \item Przeglądanie historii wypożyczeń danego klienta.
\end{itemize}

\subsubsection{Zarządzanie książkami}
\begin{figure}[H]
    \centering
    \includegraphics[width=1\linewidth]{Zrzut ekranu z 2026-01-12 12-19-32.png}
    \caption{Lista książek}
    \label{fig:books}
\end{figure}
Strona \texttt{/books} umożliwia:
\begin{itemize}
    \item Przeglądanie listy wszystkich książek w bibliotece.
    \item Dodawanie nowych książek (tytuł, autor).
    \item Edycję danych istniejących książek.
    \item Usuwanie książek.
    \item Przypisywanie kart RFID do książek.
    \item Sprawdzanie dostępności książek.
\end{itemize}

\subsubsection{Proces wypożyczania i zwracania}
System umożliwia wypożyczanie i zwracanie książek na dwa sposoby:
\begin{enumerate}
    \item \textbf{Przez terminal RFID:}
    \begin{figure}
        \centering
        \includegraphics[width=1\linewidth]{Zrzut ekranu z 2026-01-12 12-26-11.png}
        \caption{Wypożyczenie}
        \label{fig:rent}
    \end{figure}
    \begin{itemize}
        \item Zeskanuj kartę klienta (LED zmieni się na zielony).
        \item Zeskanuj kartę książki (system automatycznie utworzy wypożyczenie).
        \item Informacja o wypożyczeniu pojawi się w aplikacji webowej w czasie rzeczywistym.
    \end{itemize}
    \item \textbf{Przez interfejs webowy:}
    \begin{itemize}
        \item Wybierz klienta i książkę z listy.
        \item Kliknij przycisk "Wypożycz" lub "Zwróć".
        \item System automatycznie zaktualizuje status wypożyczenia.
    \end{itemize}
\end{enumerate}

\subsubsection{Wyświetlacz OLED na terminalu RFID}
Terminal RFID wyświetla następujące informacje na wyświetlaczu OLED:
\begin{itemize}
    \item \textbf{Oczekiwanie na kartę} - komunikat zachęcający do przyłożenia karty.
    \item \textbf{Wykryto kartę} - wyświetlenie UID zeskanowanej karty.
    \item \textbf{Przetwarzanie} - informacja o przetwarzaniu danych przez backend.
    \item \textbf{Dane klienta/książki} - wyświetlenie informacji o znalezionym kliencie lub książce.
    \item \textbf{Nowa karta} - informacja o nieznanej karcie.
    \item \textbf{Błąd} - komunikat o błędzie operacji.
\end{itemize}
\end{document}