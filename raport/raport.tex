\documentclass[12pt,a4paper]{article}

\usepackage[utf8]{inputenc}
\usepackage[T1]{fontenc}
\usepackage[polish]{babel}
\usepackage{geometry}
\usepackage{graphicx}
\usepackage{hyperref}
\usepackage{listings}
\usepackage{float}
\usepackage{caption}
\usepackage{setspace}

\geometry{margin=2.5cm}
\onehalfspacing


\title{
\textbf{Raport z realizacji projektu programistycznego} \\[0.5cm]
\large System biblioteczny
}

\author{
Autorzy (grupa C): \\[0.2cm]
Dawid Błaszczyk - nr indeksu \\
Błazej Kowal - nr indeksu \\
Alina Lenart - nr indeksu \\
Bartosz Wacławiak - nr indeksu \\[0.4cm]
Prowadzący laboratorium: \\
dr inż. Krzysztof Chudzik
}

\date{
Data ukończenia pracy: \\ 
07.01.2025
}

\begin{document}

\maketitle
\thispagestyle{empty}
\newpage

% Spis treści
\tableofcontents
\newpage

\section{Wymagania projektowe}

\subsection{Wymagania funkcjonalne}

System biblioteczny IoT musi spełniać następujące wymagania funkcjonalne:

\subsubsection{Obsługa kart RFID}
\begin{itemize}
    \item System musi umożliwiać odczyt kart RFID za pomocą czytnika MFRC522.
    \item System musi automatycznie wykrywać przyłożenie i zabranie karty RFID.
    \item Odczytany identyfikator karty (UID) musi być konwertowany do formatu szesnastkowego i przesyłany do serwera centralnego.
    \item System musi rozróżniać między kartami klientów biblioteki i kartami przypisanymi do książek.
\end{itemize}

\subsubsection{Zarządzanie bazą danych}
\begin{itemize}
    \item System musi przechowywać informacje o klientach (imię, nazwisko, powiązana karta RFID).
    \item System musi przechowywać informacje o książkach (tytuł, autor, powiązana karta RFID).
    \item System musi rejestrować wypożyczenia i zwroty książek z datami operacji.
    \item System musi umożliwiać automatyczne tworzenie nowych rekordów kart przy pierwszym skanowaniu nieznanego UID.
\end{itemize}

\subsubsection{Proces wypożyczania i zwracania}
\begin{itemize}
    \item System musi umożliwiać wypożyczenie książki poprzez zeskanowanie karty klienta, a następnie karty książki.
    \item System musi umożliwiać zwrot książki poprzez analogiczny proces skanowania.
    \item System musi wyświetlać informacje o aktywnych wypożyczeniach klienta po zeskanowaniu jego karty.
    \item System musi weryfikować poprawność operacji (np. czy książka jest dostępna, czy klient już ją wypożyczył).
\end{itemize}

\subsubsection{Interfejs użytkownika}
\begin{itemize}
    \item System musi posiadać webowy interfejs graficzny dostępny przez przeglądarkę.
    \item Interfejs musi umożliwiać przeglądanie listy wszystkich klientów, książek i wypożyczeń.
    \item Interfejs musi umożliwiać ręczne dodawanie, edycję i usuwanie klientów oraz książek.
    \item Interfejs musi wyświetlać informacje o zeskanowanej karcie w czasie rzeczywistym.
    \item Interfejs musi umożliwiać przeprowadzenie pełnego procesu wypożyczania/zwrotu z graficznym przewodnikiem.
\end{itemize}

\subsubsection{Komunikacja MQTT}
\begin{itemize}
    \item Terminale RFID (Raspberry Pi) muszą komunikować się z serwerem centralnym przez protokół MQTT.
    \item System musi publikować zdarzenia skanowania kart na topic \texttt{raspberry/rfid/scan}.
    \item Serwer musi odpowiadać z danymi o kliencie lub książce na topic \texttt{raspberry/rfid/response}.
    \item System musi obsługiwać sterowanie diodami LED przez MQTT (topic \texttt{raspberry/led}).
\end{itemize}

\subsubsection{Informacje zwrotne dla użytkownika terminala}
\begin{itemize}
    \item System musi sygnalizować gotowość do skanowania zieloną diodą LED.
    \item System musi sygnalizować przetwarzanie karty czerwoną diodą LED.
    \item System musi emitować dźwięk buzzera po pomyślnym odczytaniu karty.
    \item System musi wyświetlać informacje o stanie operacji na wyświetlaczu OLED (oczekiwanie, wykryto kartę, przetwarzanie, dane klienta/książki).
\end{itemize}

\subsection{Wymagania niefunkcjonalne}

System musi spełniać następujące wymagania niefunkcjonalne:

\subsubsection{Wydajność}
\begin{itemize}
    \item Czas odpowiedzi serwera na żądanie API nie powinien przekraczać 500ms w warunkach normalnego obciążenia.
    \item System musi przetwarzać zdarzenia RFID w czasie rzeczywistym (opóźnienie poniżej 1 sekundy od momentu skanowania do wyświetlenia informacji).
    \item Aplikacja webowa musi ładować się w czasie nie dłuższym niż 3 sekundy przy standardowym połączeniu internetowym.
\end{itemize}

\subsubsection{Niezawodność i stabilność}
\begin{itemize}
    \item System musi być odporny na tymczasową utratę połączenia z brokerem MQTT i automatycznie wznawiać komunikację.
    \item W przypadku błędu odczytu karty RFID, system musi wyświetlić komunikat o błędzie i umożliwić ponowną próbę.
    \item Baza danych musi zapewniać integralność danych (brak duplikatów wypożyczeń, prawidłowe daty operacji).
\end{itemize}

\subsubsection{Skalowalność}
\begin{itemize}
    \item Architektura systemu musi umożliwiać łatwe dodanie kolejnych terminali RFID bez modyfikacji kodu serwera.
    \item Baza danych musi być zaprojektowana w sposób umożliwiający przechowywanie tysięcy rekordów klientów i książek.
\end{itemize}

\subsubsection{Bezpieczeństwo}
\begin{itemize}
    \item Komunikacja między komponentami systemu odbywa się w sieci lokalnej (brak wymogu szyfrowania w wersji proof-of-concept).
    \item Dane w bazie danych muszą być zabezpieczone przed nieautoryzowanym dostępem poprzez odpowiednią konfigurację uprawnień.
    \item System musi walidować wszystkie dane wejściowe z API, aby zapobiec nieprawidłowym operacjom.
\end{itemize}

\subsubsection{Środowisko uruchomieniowe i przenośność}
\begin{itemize}
    \item Terminal RFID musi działać na platformie Raspberry Pi 4B z systemem operacyjnym Raspberry Pi OS.
    \item Serwer backendowy musi być uruchamialny na systemach Linux, macOS i Windows.
    \item Aplikacja webowa musi być kompatybilna z nowoczesnymi przeglądarkami (Chrome, Firefox, Safari, Edge).
    \item System musi wykorzystywać konteneryzację Docker dla brokera MQTT (eclipse-mosquitto).
    \item Baza danych SQLite musi być przenośna i nie wymagać dodatkowej konfiguracji serwera baz danych.
\end{itemize}

\subsubsection{Technologie i standardy}
\begin{itemize}
    \item Backend: Node.js z frameworkiem NestJS, TypeScript, TypeORM.
    \item Frontend: React 19 z TypeScript, React Router, Tailwind CSS, Vite.
    \item Komunikacja IoT: Protokół MQTT z brokerem Eclipse Mosquitto.
    \item Baza danych: SQLite 3.
    \item Raspberry Pi: Python 3 z bibliotekami paho-mqtt (klient MQTT) i mfrc522 (obsługa czytnika RFID).
    \item Komunikacja w czasie rzeczywistym: WebSockets (Socket.IO) dla aktualizacji interfejsu użytkownika.
\end{itemize}

\subsubsection{Użyteczność}
\begin{itemize}
    \item Interfejs graficzny musi być intuicyjny i nie wymagać specjalistycznego przeszkolenia.
    \item System musi dostarczać jasne komunikaty o stanie operacji (powodzenie, błąd, oczekiwanie).
    \item Wyświetlacz OLED na terminalu RFID musi prezentować czytelne informacje o aktualnym stanie systemu.
    \item Kolorowe diody LED muszą jednoznacznie sygnalizować stan systemu (zielony = gotowy, czerwony = przetwarzanie/błąd).
\end{itemize}

\subsubsection{Dokumentacja i kod}
\begin{itemize}
    \item Kod źródłowy musi być czytelny i zgodny ze standardami danego języka programowania.
    \item Kluczowe funkcje systemu muszą być opatrzone komentarzami wyjaśniającymi logikę działania.
    \item Projekt musi zawierać pliki konfiguracyjne umożliwiające łatwe uruchomienie systemu.
\end{itemize}

TODO ciag dalszy

\section{Opis architektury systemu}

System zostal zaprojektowany w architekturze wielowarstwowej. Sklada się z następujących elementów:
\begin{itemize}
    \item warstwy klienckiej,
    \item warstwy serwerowej,
    \item warstwy komunikacyjnej,
    \item warstwy danych.
\end{itemize}

\subsection{Schemat architektury aplikacji}
Poniżej przedstawiono schemat architektury systemu z uwzględnieniem architektury sieciowej.

\begin{figure}[H]
    \centering
   % \includegraphics[width=0.9\textwidth]{architecture.png}
    \caption{Schemat architektury systemu}
\end{figure}

\section{Opis implementacji i zastosowanych rozwiązań}

\subsection{Kluczowe elementy implementacji}
W tej sekcji opisano najważniejsze fragmenty kodu odpowiedzialne za kluczowe funkcje systemu.

\begin{lstlisting}[language=Python, caption={Przykladowy fragment kodu}]
def example_function():
    print("Przykladowa funkcja systemu")
\end{lstlisting}

\subsection{Implementacja komunikacji MQTT}
Opis zastosowanego mechanizmu komunikacji MQTT, struktury topiców oraz sposobu przesylania danych.

\begin{lstlisting}[language=Python, caption={Fragment implementacji MQTT}]
client.connect(broker_address)
client.publish("example/topic", payload)
\end{lstlisting}

\subsection{Szyfrowanie i uwierzytelnianie}
Opis zastosowanych mechanizmów zabezpieczeń:
\begin{itemize}
    \item sposób uwierzytelniania użytkowników,
    \item mechanizmy szyfrowania transmisji danych,
    \item zabezpieczenia dostępu do zasobów systemu.
\end{itemize}

\subsection{Inne istotne rozwiązania}
Opis dodatkowych elementów implementacyjnych uznanych przez Autorów za istotne.

\section{Opis dzialania i prezentacja interfejsu}

\subsection{Instalacja i uruchomienie aplikacji}
Opis kroków niezbędnych do uruchomienia aplikacji:
\begin{enumerate}
    \item Pobranie kodu źródlowego.
    \item Instalacja wymaganych zależności.
    \item Konfiguracja środowiska.
    \item Uruchomienie aplikacji.
\end{enumerate}

\subsection{Prezentacja interfejsu użytkownika}

\begin{figure}[H]
    \centering
  %  \includegraphics[width=0.8\textwidth]{ui_example.png}
    \caption{Przykladowy ekran aplikacji}
\end{figure}

Opis przedstawionych ekranów oraz sposobu interakcji użytkownika z systemem.

\section{Opis wkladu pracy Autorów}

\subsection{Autor 1 – Imię Nazwisko}
\begin{itemize}
    \item Zakres wykonanych prac.
    \item Odpowiedzialność za konkretne moduly.
    \item Szacunkowy wklad procentowy: XX\%.
\end{itemize}

\subsection{Autor 2 – Imię Nazwisko}
\begin{itemize}
    \item Zakres wykonanych prac.
    \item Odpowiedzialność za konkretne moduly.
    \item Szacunkowy wklad procentowy: XX\%.
\end{itemize}

\section{Podsumowanie}

Projekt zostal zrealizowany zgodnie z zalożonymi wymaganiami funkcjonalnymi i niefunkcjonalnymi.  
W trakcie implementacji napotkano na następujące trudności:
\begin{itemize}
    \item opis problemów technicznych,
    \item ograniczenia czasowe lub sprzętowe.
\end{itemize}

Możliwe kierunki dalszego rozwoju systemu:
\begin{itemize}
    \item rozbudowa funkcjonalności,
    \item poprawa bezpieczeństwa,
    \item optymalizacja wydajności.
\end{itemize}

\section{Literatura}

\begin{itemize}
    \item Dokumentacja użytych technologii.
    \item Artykuly techniczne.
    \item Materialy dydaktyczne.
\end{itemize}

\section{Aneks}

Kod źródlowy projektu zostal dolączony w formie elektronicznej jako zalącznik.

\end{document}
